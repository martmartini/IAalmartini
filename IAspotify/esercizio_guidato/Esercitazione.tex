\documentclass{exam}

\usepackage[utf8]{inputenc}
\usepackage[italian]{babel}
\usepackage[margin=1in]{geometry}
\usepackage{amsmath,amssymb}
\usepackage{tabularray}
\usepackage{array}
\usepackage{multirow}
\usepackage{graphicx}
\usepackage{eurosym}
\usepackage{pgfplots} 
\usepackage{multicol}
\usepackage{geometry}
\usepackage{arev}
\usepackage{siunitx}
\usepackage[
 font=normal % or small or scriptsize
 ]{subfig}
\usetikzlibrary{shapes.geometric}

\linespread{2}
\setlength\answerlinelength{7cm}

% LAYOUT FOGLIO
\geometry{a4paper, top=1cm, bottom=1cm, left=1cm, right=1cm}
\renewcommand{\arraystretch}{1.3}
\setlength{\tabcolsep}{10pt}

\begin{document}
%\input{files_struct/intestazione.tex}  \unskip 

\noindent Classe:\\
Nome e Cognome:


\noindent \rule{\textwidth}{0.4pt}
\Large{L'algoritmo k$-$NN}\\
\normalsize
Si ha a disposizione il seguente \textbf{dataset di training} con riportati i dati relativi alla playlist di un utente.\\

\begin{small}
\begin{tabular}{|c|c|c|}
\hline
Danceability & Acousticness  & Label\\ \hline
1 & 5 & Like \\
2 & 8 & Like\\
4 & 6 &  Like\\
4 & 10 &  Like\\
5 & 3 &  Like\\
5 & 4 &  Like\\
8 & 3 & Don't like\\
8 & 5 & Don't like\\
10 & 9 & Don't like\\
9 & 8 & Don't like\\
9 & 10 & Don't like\\
\hline
\end{tabular}
\end{small}

La prima \textit{feature} (colonna) rappresenta la danceability di una canzone, la seconda l'acousticness. Le prime due colonne costituiscono la mia matrice $X_{train}$.
Alll'utente della tua piattaforma solo alcune canzoni piacciono. L'informazione è riportata nella terza colonna, \textit{label}. Questa colonna è il vettore $y_{train}$ che voglio provare a predirre in base ai dati della matrice $X_{train}$.
 
Si ha poi a disposizione un secondo dataset su cui testare il modello, \textbf{dataset di test}.  \\

\begin{small}
\begin{tabular}{|c|c|c|}
\hline
Danceability & Acousticness  & Label\\ \hline
5 & 8 & Don't like \\
7 & 4 &  Like\\
10 & 4 &  Don't like\\
10 & 8 &  Don't like\\
3 & 7 & Like\\
\hline
\end{tabular}
\end{small}

\begin{questions}

\question 
Consideriamo nel \textbf{dataset di training} le due \textit{feature} come le coordinate $(x,y)$ di un punto nel piano cartesiano come riportato nel seguente grafico.


    \begin{tikzpicture}
    \begin{axis}
    [
    axis x line=middle,
    axis y line=middle,
    axis equal image, 
    width=0.8\textwidth,
    %height=3\textwidth,
    xmin=-1,
    ymin=-1, 
    xmax=11,
    ymax=11,
    samples=100,
    grid=major,
    %xtick={-1,1,2,3,4},
    %ytick={-1,1,2,3,4,5},
    xlabel={Danceability},
    ylabel={Acousticness}] 
    \node[label={[yshift=-2ex, xshift=3ex]180:{$A_1 $}},circle,fill,inner sep=2.5pt] at (axis cs: 1,5) {};
    \node[label={[yshift=-2ex, xshift=3ex]180:{$A_2 $}},circle,fill,inner sep=2.5pt] at (axis cs: 2,8) {};
    \node[label={[yshift=-2ex, xshift=3ex]180:{$A_3 $}},circle,fill,inner sep=2.5pt] at (axis cs: 4,6) {};
    \node[label={[yshift=-2ex, xshift=3ex]180:{$A_4 $}},circle,fill,inner sep=2.5pt] at (axis cs: 4,10) {};
    \node[label={[yshift=-2ex, xshift=3ex]180:{$A_5 $}},circle,fill,inner sep=2.5pt] at (axis cs: 5,3) {};
    \node[label={[yshift=-2ex, xshift=3ex]180:{$A_6 $}},circle,fill,inner sep=2.5pt] at (axis cs: 5,4) {};
%    \node[label={[yshift=-1ex, xshift=3ex]180:{$ $}},circle,fill,inner sep=3pt] at (axis cs: 7,4) {};
%    \node[label={[yshift=-2ex, xshift=3ex]180:{$ $}},mark=*,fill,inner sep=3pt] at (axis cs: 5,8) {};
    \node[label={[yshift=-2ex, xshift=3ex]180:{$B_1$}},rectangle,fill,inner sep=3pt] at (axis cs: 8,3) {};
    \node[label={[yshift=-2ex, xshift=3ex]180:{$B_2 $}},rectangle,fill,inner sep=3pt] at (axis cs: 8,5) {};
%    \node[label={[yshift=-2ex, xshift=3ex]180:{$ $}},mark=x,fill,inner sep=3pt] at (axis cs: 10,4) {};
    \node[label={[yshift=-2ex, xshift=3ex]180:{$B_3 $}},rectangle,fill,inner sep=3pt] at (axis cs: 10,9) {};
    \node[label={[yshift=-2ex, xshift=3ex]180:{$B_4 $}},rectangle,fill,inner sep=3pt] at (axis cs: 9,8) {};
        \node[label={[yshift=-2ex, xshift=3ex]180:{$B_5$}},rectangle,fill,inner sep=3pt] at (axis cs: 9,10) {};
    \end{axis}
    \end{tikzpicture}

Classifica i seguenti punti $P_1(5;8),\, P_2(7,4), P_3(10,4), P_4(10,8), P_5(3,7)$ considerando $k=3$.

\begin{parts}
\part Come si calcola la distanza tra due punti non allineati orizzontalmente o verticalmente?\\[0.3cm]
\makebox[0.7\textwidth]{\dotfill}
\part E nel caso di punti allineati orizzontalmente o verticalmente?\\[0.3cm]
\makebox[0.7\textwidth]{\dotfill}
\part Riporta le distanze nella seguente tabella

\begin{tabular}{|*{12}{m{0.8cm}|}}
\hline
 & $A_1 $ & $A_2 $ & $A_3 $ & $A_4 $ & $A_5 $ & $A_6 $ & $B_1 $ & $B_2 $ & $B_3 $ & $B_4 $ & $B_5$\\
\hline
$P_1$ & & & & & & & & & &  &\\
\hline
$P_2$  & & & & & & & & & & & \\
\hline
$P_3$  & & & & & & & & & & & \\
\hline
$P_4$  & & & & & & & & & & & \\
\hline
$P_5$ & & & & & & & & & &  &\\
\hline
\end{tabular}

\part Ordinare le distanze in ordine crescente, calcolare la frequenza per ogni label e classificare il punto.

\begin{tabular}{|*{7}{m{1.5cm}|}}
\hline
 &  \multicolumn{3}{|c|}{Distanze ordinate} &$f_{ \text{Like}}$&$f_{\text{Don't like}}$& label \\
\hline
$P_1$  &  &  &  &  &  & \\
\hline
$P_2$  &  &  &  &  &  & \\
\hline
$P_3$  &  &  &  &  &  & \\
\hline
$P_4$  &  &  &  &  &  & \\
\hline
$P_5$  &  &  &  &  &  &\\
\hline
\end{tabular}

\end{parts}


\question 
Sai che i cinque punti che hai classificato precedentemente in realtà sono rispettivamente Don't like, Like, Don't like, Don't like, Like.

\begin{parts}

\part Completare la seguente matrice di confusione


\begin{tabular}{c|m{2cm}|m{2cm}|}
\multicolumn{1}{c}{} & \multicolumn{2}{c}{Valori predetti} \\ \cline{2-3}
\multirow{2}{*}{Valori reali} &  &  \\ \cline{2-3}
&  &  \\ \cline{2-3}
\end{tabular}

\part Come si calcola l'accuratezza?\\[0.3cm]
\makebox[0.7\textwidth]{\dotfill}

\part In questo caso l'accuratezza risulta \makebox[2cm]{\dotfill}
 
\end{parts}
\question Ponendo k=5

\begin{parts}
\part   il punto $P_2$ come viene classificato? Riportare i conti svolti.
\makeemptybox{6cm}


\part Come cambia la matrice di confusione? E l'accuratezza?
\begin{tabular}{c|m{2cm}|m{2cm}|}
\multicolumn{1}{c}{} & \multicolumn{2}{c}{Valori predetti} \\ \cline{2-3}
\multirow{2}{*}{Valori reali} &  &  \\ \cline{2-3}
&  &  \\ \cline{2-3}
\end{tabular}

In questo caso l'accuratezza risulta \makebox[2cm]{\dotfill}

\end{parts}
% P_1 quadrati, P_2 cerchio P_3 quadrato






\end{questions}
\end{document}
